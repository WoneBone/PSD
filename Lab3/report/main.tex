\documentclass[12pt]{article}

% ----------------------------------------------------------------------
% Definir packages externos, língua, margens, tipos de letra, novos 
% comandos e cores
% ----------------------------------------------------------------------
\usepackage[utf8]{inputenc} % Codificação utilizada
\usepackage[english]{babel} % Idioma de escrita

\usepackage[export]{adjustbox} % Alinhar imagens
\usepackage{amsmath} % Comandos extra para escrita matemática
\usepackage{amssymb} % Símbolos matemáticos
\usepackage{anysize} % Personalizar as margens
    \marginsize{2cm}{2cm}{2cm}{2cm} % {esquerda}{direita}{cima}{baixo}
\usepackage{appendix} % Apêndices
\usepackage{cancel} % Cancelar expressões
\usepackage{caption} % Legendas
    \DeclareCaptionFont{newfont}{\fontfamily{cmss}\selectfont}
    \captionsetup{labelfont={bf, newfont}}
%\usepackage{cite} % Citações, tipo [1 - 3]
\usepackage{color} % Colorir texto
\usepackage{fancyhdr} % Cabeçalho e rodapé
    \pagestyle{fancy}
    \fancyhf{}
    \fancyhead[L]{\footnotesize\fontfamily{cmss}\selectfont IST} % Esquerda do cabeçalho
    \fancyhead[R]{\footnotesize\fontfamily{cmss}\selectfont ULisboa} % Direita do cabeçalho
    \fancyfoot[L]{\footnotesize\fontfamily{cmss}\selectfont Projeto de Sistemas Digitais} % Esquerda do rodapé
    \fancyfoot[C]{\thepage} % Centro do rodapé
    \fancyfoot[R]{\footnotesize\fontfamily{cmss}\selectfont LEEC} % Direita do rodapé
    \renewcommand{\footrulewidth}{0.4pt} % Régua do rodapé
\usepackage{float} % Utilizar o especificador [H] nas figuras
\usepackage{graphicx} % Imagens em LaTeX
\usepackage[colorlinks = true, plainpages = true, linkcolor = istblue, urlcolor = istblue, citecolor = istblue, anchorcolor = istblue]{hyperref}
\usepackage{indentfirst} % Primeiro parágrafo
\usepackage{siunitx} % Unidades SI
\usepackage{subcaption} % Subfiguras
\usepackage{titlesec} % Tipo de letra
    \titleformat{\section}{\fontfamily{cmss}\selectfont\Large\bfseries}{\thesection}{1em}{}
    \titleformat{\subsection}{\fontfamily{cmss}\selectfont\large\bfseries}{\thesubsection}{1em}{}
    \titleformat{\subsubsection}{\fontfamily{cmss}\selectfont\normalsize\bfseries}{\thesubsubsection}{1em}{}
    \fancyfoot[C]{\fontfamily{cmss}\selectfont\thepage}

% Encher de texto aleatório (apagar)
\usepackage{lipsum}
\usepackage{duckuments}
\usepackage[table]{xcolor}

% Novos e renovar comandos
\newcommand{\sen}{\operatorname{\sen}} % Definição da função seno
\newcommand{\HRule}{\rule{\linewidth}{0.5mm}} % Definição de uma régua
\renewcommand{\appendixpagename}{\LARGE \fontfamily{cmss}\selectfont Apêndices}
\renewcommand{\appendixtocname}{Apêndices}

% Cores
\definecolor{istblue}{RGB}{3, 171, 230}
\definecolor{dkgreen}{rgb}{0,0.6,0}
\definecolor{gray}{rgb}{0.5,0.5,0.5}

%importar bibliografia
\usepackage[
backend=biber,
style=alphabetic,
sorting=nyt,
]{biblatex}
\addbibresource{refs.bib} 

%%%%%%%%%%%%%%%%%%%%%%%%%%%%%%%%%%%%%%%%%%%%%%%%%%%%%%%%%%%%%%%%%%%%%%%%
%                               Documento                              %
%%%%%%%%%%%%%%%%%%%%%%%%%%%%%%%%%%%%%%%%%%%%%%%%%%%%%%%%%%%%%%%%%%%%%%%%
\begin{document}

% ----------------------------------------------------------------------
% Capa
% ----------------------------------------------------------------------
\begin{center}
    \begin{figure}
        \vspace{-1.0cm}
        \includegraphics[scale = 0.3, left]{images/IST_A.eps} % Tipo de assinatura do IST
    \end{figure}
    \mbox{}\\[2.0cm]
    \textsc{\Huge Projeto de Sistemas Digitais}\\[2.5cm]
    \textsc{\LARGE MEEC}\\[2.0cm]
    \HRule\\[0.4cm]
    {\large \bf {\fontfamily{cmss}\selectfont Complex Determinant Computation} [\texttt{EN}]}\\[0.2cm]
    \HRule\\[1.5cm]
\end{center}

\begin{flushleft}
    \textbf{\fontfamily{cmss}\selectfont Authors:}
\end{flushleft}

\begin{center}
    \begin{minipage}{0.4\textwidth}
        \begin{flushleft}
            Alexandre Santos (99884)\\
            Nuno Abreu (103416)\\
            Carlos Reis (103166) \\
        \end{flushleft}
    \end{minipage}%
    \begin{minipage}{0.6\textwidth}
        \begin{flushright}
            \href{mailto:ares.santos@tecnico.ulisboa.pt}{\texttt{ares.santos@tecnico.ulisboa.pt}}\\
            \href{mailto:nuno.g.tribolet.de.abreu@tecnico.ulisboa.pt}{\texttt{nuno.g.tribolet.de.abreu@tecnico.ulisboa.pt}}\\
            \href{mailto:guilherme.garcia@tecnico.ulisboa.pt}{\texttt{guilherme.garcia@tecnico.ulisboa.pt}}\\
            
        \end{flushright}
    \end{minipage}


\end{center}
    
\begin{flushleft}
    \large $\boxed{\text{\bf \fontfamily{cmss}\selectfont Group 10}}$\\[4.0cm]
\end{flushleft}
    
\begin{center}
    \large \bf \fontfamily{cmss}\selectfont 2024/2025 -- 1st Semester, P1
\end{center}

\thispagestyle{empty}

\setcounter{page}{0}

\newpage

% ----------------------------------------------------------------------
% Conteúdo
% ----------------------------------------------------------------------


\section{Introduction}
The following report pertains to the third laboratory task of Project of Digital Systems.

The objective is to design a circuit that computes determinants of complex valued matrices, with VHDL hardware description language and making use of simulation and logic synthesis. The circuit is then implemented with the AMD Vivado™ tool in the Digilent Basys 3™ FPGA board. 

\section{Data-flow Graph}
To start the work a data-flow graph was made accordingly with the operations needed to compute the determinant's. The result is presented in image \ref{fig:dataflow} corresponds to one loop iteration of the algorithm. This graph shows the serialization limitation of the sequence of operations needed and allows us to make a priority list based on the critical path.

\begin{figure}[H]
	\centering
	\includegraphics[width=0.35\linewidth]{images/DataFlowGraph.png}
	\caption{Data-Flow Graph}
	\label{fig:dataflow}
\end{figure}

\begin{figure}[H]
	\centering
	\includegraphics[width=0.35\linewidth]{images/DataPath.png}
	\caption{Data-Flow Graph}
	\label{fig:datapath}
\end{figure}


\begin{figure}[H]
	\centering
	\includegraphics[width=0.55\linewidth]{images/FSM_lab3.png}
	\caption{Finite State Machine}
	\label{fig:fsm}
\end{figure}

\section{Circuit Design}
\subsection{Data-Path}



\subsection{Control Unit}
The control unit implements a finite state machine with 14 states. A start state awaits for the command to begin calculations(BTNR) , followed by 5 states to fill up the pipeline, 4 states that repeat until the last matrix arrives at the datapath, 3 states to empty it and one done state where the machine and presents the results.
Two counters keep trace of input and output memory addresses.

Each state takes up one clock cycle, except for the start and done states which extend indefinitely. A reset signal returns the control to the start state, the counters to 0 and all registers to 0.


For simplicity, we will only show the simulations done for MemIn1.

The test-bench used is very simple. All it does is reset the circuit followed by handling the clock.

The clock cycle of the circuit was set to 18ns in the constraints and the same value was used for the test-bench. This left us with a WNS of 3.697ns.


Figure is a portion of the behavioral simulation of the circuit. We can observe the different states and the data out change as the calculations finish and write enable toggles. Notice also how this is address 3 of the memory, the final value of -536804112 is correct.


Figure is a similar cutout, but now from the Post-Implementation Timing Simulation. We now can't see the states, however we can see that the write enable and data out behaviour is the same. Notice how the result of -343976512 is correct for address 2.


\section{Conclusion}
Through this laboratory we successfully simulated and implemented a datapath, control unit(with two counters for address tracking) components and memory access.

The focus of this project was to design a circuit in an efficient way to compute various determinants. This was achieved through pipelining.
  
We implemented the final circuit to test if its behaviour was as designed and verified its outputs with the provided script. This allowed us to
find errors in our design, correct them in timely manner and fine tune the clock period to a reasonable value maintaining a positive Worst Negative Slack.

Further optimizations were discussed such as pipelining and reducing the number of registers. Both solutions implied a trade-off, the former requiring more components and the latter increasing the clock period thus decreasing throughput.

\printbibliography
\end{document}
