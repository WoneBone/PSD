\documentclass[12pt]{article}

% ----------------------------------------------------------------------
% Definir packages externos, língua, margens, tipos de letra, novos 
% comandos e cores
% ----------------------------------------------------------------------
\usepackage[utf8]{inputenc} % Codificação utilizada
\usepackage[english]{babel} % Idioma de escrita

\usepackage[export]{adjustbox} % Alinhar imagens
\usepackage{amsmath} % Comandos extra para escrita matemática
\usepackage{amssymb} % Símbolos matemáticos
\usepackage{anysize} % Personalizar as margens
    \marginsize{2cm}{2cm}{2cm}{2cm} % {esquerda}{direita}{cima}{baixo}
\usepackage{appendix} % Apêndices
\usepackage{cancel} % Cancelar expressões
\usepackage{caption} % Legendas
    \DeclareCaptionFont{newfont}{\fontfamily{cmss}\selectfont}
    \captionsetup{labelfont={bf, newfont}}
\usepackage{cite} % Citações, tipo [1 - 3]
\usepackage{color} % Colorir texto
\usepackage{fancyhdr} % Cabeçalho e rodapé
    \pagestyle{fancy}
    \fancyhf{}
    \fancyhead[L]{\footnotesize\fontfamily{cmss}\selectfont IST} % Esquerda do cabeçalho
    \fancyhead[R]{\footnotesize\fontfamily{cmss}\selectfont ULisboa} % Direita do cabeçalho
    \fancyfoot[L]{\footnotesize\fontfamily{cmss}\selectfont Projeto de Sistemas Digitais} % Esquerda do rodapé
    \fancyfoot[C]{\thepage} % Centro do rodapé
    \fancyfoot[R]{\footnotesize\fontfamily{cmss}\selectfont LEEC} % Direita do rodapé
    \renewcommand{\footrulewidth}{0.4pt} % Régua do rodapé
\usepackage{float} % Utilizar o especificador [H] nas figuras
\usepackage{graphicx} % Imagens em LaTeX
\usepackage[colorlinks = true, plainpages = true, linkcolor = istblue, urlcolor = istblue, citecolor = istblue, anchorcolor = istblue]{hyperref}
\usepackage{indentfirst} % Primeiro parágrafo
\usepackage{siunitx} % Unidades SI
\usepackage{subcaption} % Subfiguras
\usepackage{titlesec} % Tipo de letra
    \titleformat{\section}{\fontfamily{cmss}\selectfont\Large\bfseries}{\thesection}{1em}{}
    \titleformat{\subsection}{\fontfamily{cmss}\selectfont\large\bfseries}{\thesubsection}{1em}{}
    \titleformat{\subsubsection}{\fontfamily{cmss}\selectfont\normalsize\bfseries}{\thesubsubsection}{1em}{}
    \fancyfoot[C]{\fontfamily{cmss}\selectfont\thepage}

% Encher de texto aleatório (apagar)
\usepackage{lipsum}
\usepackage{duckuments}
\usepackage[table]{xcolor}

% Novos e renovar comandos
\newcommand{\sen}{\operatorname{\sen}} % Definição da função seno
\newcommand{\HRule}{\rule{\linewidth}{0.5mm}} % Definição de uma régua
\renewcommand{\appendixpagename}{\LARGE \fontfamily{cmss}\selectfont Apêndices}
\renewcommand{\appendixtocname}{Apêndices}

% Cores
\definecolor{istblue}{RGB}{3, 171, 230}
\definecolor{dkgreen}{rgb}{0,0.6,0}
\definecolor{gray}{rgb}{0.5,0.5,0.5}

%%%%%%%%%%%%%%%%%%%%%%%%%%%%%%%%%%%%%%%%%%%%%%%%%%%%%%%%%%%%%%%%%%%%%%%%
%                               Documento                              %
%%%%%%%%%%%%%%%%%%%%%%%%%%%%%%%%%%%%%%%%%%%%%%%%%%%%%%%%%%%%%%%%%%%%%%%%
\begin{document}

% ----------------------------------------------------------------------
% Capa
% ----------------------------------------------------------------------
\begin{center}
    \begin{figure}
        \vspace{-1.0cm}
        \includegraphics[scale = 0.3, left]{Imagens/IST_A.eps} % Tipo de assinatura do IST
    \end{figure}
    \mbox{}\\[2.0cm]
    \textsc{\Huge Projeto de Sistemas Digitais}\\[2.5cm]
    \textsc{\LARGE MEEC}\\[2.0cm]
    \HRule\\[0.4cm]
    {\large \bf {\fontfamily{cmss}\selectfont Computing Determinants : Scheduling and Resource Sharing} [\texttt{EN}]}\\[0.2cm]
    \HRule\\[1.5cm]
\end{center}

\begin{flushleft}
    \textbf{\fontfamily{cmss}\selectfont Authors:}
\end{flushleft}

\begin{center}
    \begin{minipage}{0.4\textwidth}
        \begin{flushleft}
            Alexandre Santos (99884)\\
            Nuno Abreu (103416)\\
            Carlos Reis nº103166 \\
        \end{flushleft}
    \end{minipage}%
    \begin{minipage}{0.6\textwidth}
        \begin{flushright}
            \href{mailto:ares.santos@tecnico.ulisboa.pt}{\texttt{ares.santos@tecnico.ulisboa.pt}}\\
            \href{mailto:nuno.g.tribolet.de.abreu@tecnico.ulisboa.pt}{\texttt{nuno.g.tribolet.de.abreu@tecnico.ulisboa.pt}}\\
            \href{mailto:guilherme.garcia@tecnico.ulisboa.pt}{\texttt{guilherme.garcia@tecnico.ulisboa.pt}}\\
            
        \end{flushright}
    \end{minipage}


\end{center}
    
\begin{flushleft}
    \large $\boxed{\text{\bf \fontfamily{cmss}\selectfont Group 10}}$\\[4.0cm]
\end{flushleft}
    
\begin{center}
    \large \bf \fontfamily{cmss}\selectfont 2024/2025 -- 1st Semester, P1
\end{center}

\thispagestyle{empty}

\setcounter{page}{0}

\newpage

% ----------------------------------------------------------------------
% Conteúdo
% ----------------------------------------------------------------------
\tableofcontents 

\newpage

\section{Introduction}
The following report pertains to the first laboratory task of Project of Digital Systems.

The objective is to design a program that computes determinants utilizing
scheduling and resource Sharing, with VHDL hardware description language and making use of simulation and logic synthesis. 

\section{Data-flow Graph}
To start the work a data-flow graph was made accordingly with the operations needed to compute the determinant's. The result is presented in image \ref{fig:data} corresponds to one loop iteration of the algorithm. This graph shows the serialization limitation of the sequence of operations needed and allows us to make a priority list based on the critical path.

\begin{figure}[H]
	\centering
	\includegraphics[width=0.35\linewidth]{Imagens/flowdat.drawio.png}
	\caption{Data-Flow Graph}
	\label{fig:data}
\end{figure}

\section{Priority List}
Using the critical path as metric the priority list was made and is represented in \ref{tab:list}. Utilizing this resource we can a choose of scheduling process that depends on the importance of each process and the resources available.

\begin{table}[H]
	\center
	\begin{tabular}{|c|c|c|}
		\hline
		  Nº & Operation & Priority \\
		\hline
		1 & mul & 3 \\
		\hline
        2 & mul & 4 \\
		\hline
        3 & add & 4 \\
		\hline
        4 & add & 3 \\
		\hline
        5 & mul & 3 \\
		\hline
        6 & mul & 2 \\
		\hline
        7 & add & 3 \\
		\hline
        8 & mul & 2 \\
		\hline
        9 & sub & 1 \\
		\hline
	\end{tabular}
    \begin{tabular}{|c|c|c|}
		\hline
		  Nº & Operation & Priority \\
		\hline
        2 & mul & 4 \\
        \hline
        3 & add & 4 \\
		\hline
        1 & mul & 3 \\
		\hline
        4 & add & 3 \\
		\hline
        5 & mul & 3 \\
		\hline
        7 & add & 3 \\
        \hline
        6 & mul & 2 \\
		\hline
        8 & mul & 2 \\
		\hline
        9 & sub & 1 \\
		\hline
	\end{tabular}
	\caption{Priority List}
	\label{tab:list}
\end{table}
\section{List Scheduling}
With the information obtained from the Priority List \ref{tab:list}. Considering that each operation requires a clock cycle and the only resources available (for arithmetic operations) are: 2 multipliers and 1 ALU (addition and subtraction). The list scheduling was done as can be seen in image \ref{fig:sch}, having all the constraints in mind.

\begin{figure}[H]
	\centering
	\includegraphics[width=0.6\linewidth]{Imagens/stop.drawio.png}
	\caption{Data-Flow Graph}
	\label{fig:sch}
\end{figure}

\end{document}